% !TeX encoding = UTF-8
% !TeX spellcheck = <none>
\section*{摘\hspace{\ccwd}要}
\addcontentsline{toc}{section}{摘要}

硕博士学位论文是研究生从事科研工作成果的主要体现。它能够集中表明学位申请者在研究工作中获得的新发明、理论或见解,是研究生申请硕士或博士学位的重要依据,也是科研领域中的重要文献资料和社会的宝贵财富。
摘要需作者简要介绍本论文的主要内容,主要为本人所完成的工作和创新点。
注:摘要的撰写应符合GB/T 7713.1-2006 的规定。摘要应具有独立性和自含性,即不阅读论文的全文,就能获得必要的信息。摘要的内容应包含与论文等同量的主要信息,供读者确定有无必要阅读全文,也可供二次文献采用。摘要一般应说明研究工作的目的、方法、结果和结论等,重点是结果和结论。摘要中应尽量避免采用图、表、化学结构式、非公知公用的符号和术语。


\vspace{12bp}
\noindent\textbf{关键词:}学位论文;论文格式;规范化;模板

\pagebreak
\section*{ABSTRACT}
\addcontentsline{toc}{section}{ABSTRACT}

Master and doctoral dissertations are the main embodiment of postgraduates engaged in scientific research. It can centrally indicate the new inventions, theories, or insights obtained by degree applicants in their research work. It is an important basis for graduate students to apply for master's or doctor's degree, and also an important literature in the field of scientific research and valuable wealth of society.
The abstract requires author to briefly introduce the main content of dissertation, mainly for the author’s own work and innovation.  
Note: The abstract should be written in accordance with the provisions of GB/T 7713.1-2006. It should be independent and self-contained, that is, necessary information can be obtained without reading the full text. Abstract content should contain the same amount of main information as the dissertation, so that readers can determine whether it is necessary to read the full text, and it can also be used for secondary literature. The abstract should generally describe the purpose, methods, results and conclusions of research work, focusing on the results and conclusions. And try to avoid using figures, tables, chemical structural formulas, non-public symbols and terminology.


\vspace{12bp}
\noindent\textbf{Keywords}:\enskip Dissertation; Dissertation format; Standardization; Template
