% !TeX spellcheck = <none>
\section{交叉引用}\label{sec:ref}
%%%%%%%%%%%%%%%%%%%%%%%%%%%%%%%%%%%%%%%%%%%%%%%%%%%%%%%
交叉引用可以说是\LaTeX的核心竞争力了。
我们经常需要在论文中引用文献和文章中的图表,比如说:“根据文献
~\cite{shen2017label},~\cite{shen2016object}和~\cite{shen2017deepskeleton}
所描述的的方法,
以式\ref{eq:cases}作为评价标准,我们可以得到如图~\ref{fig1}所示的性能曲线以及
表~\ref{tab:sk-fmeasure}中的定量型能比较。从图~\ref{fig1}和表~\ref{tab:sk-fmeasure}的结果来看,~\cite{shen2016object}和~\cite{shen2017deepskeleton}
具有较好的检测效果”。

如果你使用Word撰写学位论文,可以想象一下情景:你的论文有50+条引用,
你要在论文中反复交叉引用这些参考文献;然后现在你发现你的绪论部分需要补充一条
参考文献,而有的引用格式要求参考文献引用标号按文中出现先后的顺序排列,
当插入一条参考文献之后你如何处理后续的参考文献编号?
%%


或者有下面一个场景:当你完成第三章写作之后发现图3.6和图3.7之间要再插入一张图,然后
你发现图3.7和图3.7之后的所有图片的标号都要改,而且你的文中所有引用到这些图的地方都需要修改。

\textbf{\LaTeX强大的交叉引用功能}将把你从繁琐的文献/图表/公式标号中解放出来,
你只用关注写作本身,其他的事情会帮你自动完成。
%
当你写完一个图表/公式,给它添加一个label属性,然后在需要引用的地方使用ref\{the-label\}
进行引用,\LaTeX将自动为你排好序号。
%
比如"根据文献~\cite{shen2017label},~\cite{shen2016object}
和~\cite{shen2017deepskeleton}所描述的的方法,
以式\ref{eq:matrix}作为评价标准,我们可以得到如图~\ref{fig1}所示的性能曲线以及
表~\ref{tab:sk-fmeasure}中的定量型能比较。从图~\ref{fig1}和表~\ref{tab:sk-fmeasure}的结果来看,~\cite{shen2016object}和~\cite{shen2017deepskeleton}
具有较好的检测效果"。

\TeX文档中的所有内容都可以添加label属性从而进行交叉引用。比如说文章的一个子章节
(subsection)就可以被引用:第\ref{sec:location}章描述了如何对\TeX元素进行定位。

更强大的是,所有的生成的引用标号都是可以点击的,
当你在生成的pdf中点击引用标号,将自动弹到对应的文献/图表/公式处。
%
另外,在文章最后的参考文献列表中,每一条参考文献的末尾都会标注这条参考文献在哪一页被引用。

%

\pagebreak
%%%%%%%%%%%%%%%%%%%%%%%%%%%%%%%%%%%%%%%%%%%%%%%%%%%%%%%
\section{有用的链接}
%%%%%%%%%%%%%%%%%%%%%%%%%%%%%%%%%%%%%%%%%%%%%%%%%%%%%%%
\begin{itemize}
	\item 数学符号速查表 \url{http://web.ift.uib.no/Teori/KURS/WRK/TeX/symALL.html}
	\item 字体大小 \url{https://texblog.org/2012/08/29/changing-the-font-size-in-latex/}
	\item 一个比较全的 \LaTeX \ WiKi \url{https://en.wikibooks.org/wiki/LaTeX}
\end{itemize}

\pagebreak
%%%%%%%%%%%%%%%%%%%%%%%%%%%%%%%%%%%%%%%%%%%%%%%%%%%%%%%
\section{参考文献}
%%%%%%%%%%%%%%%%%%%%%%%%%%%%%%%%%%%%%%%%%%%%%%%%%%%%%%%
\LaTeX使用bib(或者latexbib)管理参考文献。新增参考文献条目时只需要在将bib格式的参考文献加入bib文件中,然后重新编译即可。在文中使用cite\{citationA\}引用即可。
点击参考文献编号\cite{shen2017deepskeleton}可跳转至对应的参考文献条目。